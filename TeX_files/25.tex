\section{Двухлучевые интерферометры. Интерферометры Рэлея, Маха-Цендера, Майкельсона. Применение интерферометров в научных исследованиях и технике: измерение малых смещений, изучение состояния поверхности, рефрактометрия (изменение показателя преломления). }
	\subsection{Двухлучевые интерферометры.}
	\begin{definition}
		Интерферометром называется измерительный прибор основным действие которого основано на явлении интерференции.  Принцип действия интерферометра заключается в следующем: пучок электромагнитного излучения с помощью того или иного устройства пространственно разделяется на два или большее количество когерентных пучков. Каждый из пучков проходит различные оптические пути и направляется на экран, создавая интерференционную картину, по которой можно установить разность фаз интерферирующих пучков в данной точке картины.
	\end{definition}
	\begin{definition}
		Двулучевыми называются интерферометры использующие в своей работе два когерентных пучка, идущих разными оптическими путями.
	\end{definition}
\subsection{Интерферометр Рэлея}
\begin{figure}[H]
	\centering
	\begin{tikzpicture}[>=latex']
	\draw[red,thick,->] (-2.9,0) -- (-2.5,0);
	\draw[red,thick,->] (-2.94,0.04) -- (-2.55,0.2);
	\draw[red,thick,->] (-2.94,-0.04) -- (-2.55,-0.2);
	\filldraw[fill=red!44!white,draw=black] (-3,0) circle (0.1) node[left] {\scriptsize{Источник}}; 
	\draw[black!77!red,thick] (-2.45,1) -- (-2.45,-1);
	\draw[white,thick] (-2.45,0.05) -- (-2.45,-0.05);
	\filldraw[fill=blue!11!white,draw=black] (-2,-0.5) arc (210:150:1) -- (-1.95,0.5) arc (30:-30:1) -- cycle;
	\draw[black!77!red,thick] (-1.5,1.2) -- (-1.5,-1.2);
	\draw[white,thick] (-1.5,0.35) -- (-1.5,0.45);
	\draw[white,thick] (-1.5,-0.35) -- (-1.5,-0.45);
	\filldraw[fill=green!11!white,draw=black, thick] (-1.2,0.2) -- (-1.2,0.6) -- (0.5,0.6) -- (0.5,0.2) -- cycle node[above, pos=0.5]{\scriptsize $n_{1}$};
	\filldraw[fill=green!33!white,draw=black, thick] (-1.2,-0.2) -- (-1.2,-0.6) -- (0.5,-0.6) -- (0.5,-0.2) -- cycle node[below, pos=0.5]{\scriptsize $n_{2}$};
	\filldraw[fill=blue!11!white,draw=black] (0.7,-0.15) -- (0.7,-0.65) -- (0.85,-0.65) -- (0.85,-0.15) -- cycle;
	\filldraw[fill=blue!11!white,draw=black] (0.7,0.15) -- (0.7,0.65) -- (0.85,0.65) -- (0.85,0.15) -- cycle;
	\filldraw[fill=blue!11!white,draw=black] (1.55,-0.5) arc (210:150:1) -- (1.6,0.5) arc (30:-30:1) -- cycle;
	\filldraw[fill=gray!44!white,draw=black] (2.2,0.3) -- (2.2,-0.3) -- (3.2,-0.3) -- (3.2,0.3) -- cycle;
	\node at (3.9,0) {\scriptsize{Детектор}};
	\draw[red] (-2.45,0) -- (-1.925,0.4);
	\draw[red] (-2.45,0) -- (-1.925,-0.4);
	\draw[red,->] (-2.45,0) -- (-2.2,0.2);
	\draw[red,->] (-2.45,0) -- (-2.2,-0.2);
	\draw[red] (-1.925,0.4) -- (-0.55,0.4);
	\draw[red] (-1.925,-0.4) -- (-0.55,-0.4);
	\draw[red,->] (-1.95,0.4) -- (-0.9,0.4);
	\draw[red,->] (-1.95,-0.4) -- (-0.9,-0.4);
	\draw[red] (-0.2,0.4) -- (1.625,0.4);
	\draw[red] (-0.2,-0.4) -- (1.625,-0.4);
	\draw[red,->] (-0.2,0.4) -- (1.3,0.4);
	\draw[red,->] (-0.2,-0.4) -- (1.3,-0.4);
	\draw[red] (1.625,0.4) -- (2.2,0);
	\draw[red] (1.625,-0.4) -- (2.2,0);
	\draw[red,->] (1.625,0.4) -- (1.9125,0.2);
	\draw[red,->] (1.625,-0.4) -- (1.9125,-0.2);
	\end{tikzpicture}
	\caption{Интерферометр Рэлея}
	\label{rel}
\end{figure}
Интерферометр Рэлея представлен на рис.~\ref{rel}.
Свет от источника пропускается через линзу, создающую параллельный пучок, и апертуры, вырезающие из него два луча (плечи интерферометра). Каждый из лучей проходит сквозь собственную кювету с газом со своим показателем преломления ($n_{1}$ и $n_{2}$). На выходе схемы расположена линза, сводящая оба пучка вместе для получения интерференционных полос в её фокусе.

Для измерений в одно из плеч вносится компенсатор — например, стеклянная пластинка, с помощью поворота которой можно изменять оптическую длину пути луча в плече. Если показатель преломления в одном из плеч равен $n_{1}$, то второй неизвестный показатель преломления равен
\begin{equation*}
n_{2} = n_{1} + \frac{\lambda_{0}}{\ell}\Delta m
\end{equation*}
где $\ell$ -- длина кюветы с газом, $\lambda _{0}$ -- длина волны источника света, $\Delta m$ -- порядок интерференции (количество пересекающихся в заданной точке интерференционных полос). При типичных параметрах установки — длине кювет в один метр, длине волны в 550 нм и порядке интерференции 1/40, — можно измерить разницу показателей преломления, равную $10^{-8}$. Чувствительность интерферометра определяется длиной кюветы. Её максимальная длина, как правило, определяется техническими возможностями контроля за температурой, так как тепловые флуктуации будут искажать показатели преломления газов.
	\subsection{Интерферометр Маха-Цендера}
	\begin{figure}[H]
		\centering
		\begin{tikzpicture}[>=latex']
		\filldraw[fill=gray!44!white,draw=black] (-3,-2.1) -- (-2.5,-2.1) -- (-2.5,-1.9) -- (-3,-1.9) -- cycle node[below] {\scriptsize{Источник}}; 
		\draw[blue!22!white,very thick] (-1.7,-2.2) -- (-1.3,-1.8);
		\draw[blue!22!white,very thick] (1.4,-1.8) -- (1,-2.2);
		\draw[black!66!white,very thick] (1.43,-1.83) -- (1.03,-2.23);
		\draw[blue!22!white,very thick] (1.0,0.5) -- (1.4,0.9);
		\draw[blue!22!white,very thick] (-1.7,0.5) -- (-1.3,0.9);
		\draw[black!66!white,very thick] (-1.73,0.53) -- (-1.33,0.93);
		\filldraw[fill=gray!44!white,draw=black] (1.0,1.6) -- (1.4,1.6) -- (1.4,1.3) -- (1.0,1.3) -- cycle node[above] {\scriptsize{Детектор}};
		\filldraw[fill=gray!44!white,draw=black] (2.0,-2.2) -- (2.3,-2.2) -- (2.3,-1.8) -- (2.0,-1.8) -- cycle node[below] {\scriptsize{Детектор}};
		\draw[red] (-2.5,-2) -- (2.0,-2);
		\draw[red,->] (-2.5,-2) -- (-2.0,-2);
		\draw[red,->] (-2.5,-2) -- (0.0,-2);
		\draw[red,->] (-2.5,-2) -- (1.75,-2);
		\draw[red] (1.2,-2) -- (1.2,1.3);
		\draw[red,->] (1.2,-2) -- (1.2,0);
		\draw[red] (-1.5,-2) -- (-1.5,0.7);
		\draw[red,->] (-1.5,-2) -- (-1.5,0.0);
		\draw[red] (-1.5,0.7) -- (1.2,0.7);
		\draw[red,->] (-1.5,0.7) -- (0,0.7);
		\draw[red,->] (1.2,0.7) -- (1.2,1.1);
		\end{tikzpicture}
		\caption{Схема интерферометра Маха-Цендера}
		\label{mach}
	\end{figure}
	Интерферометр Маха-Цендера устроен следующим образом (Рис.~(\ref{mach})).
	На входе интерферометра находится полупрозрачное зеркало, расщепляющее световой поток на два луча. Они сводятся вместе после отражения от двух непрозрачных зеркал в четвёртом зеркале. Зеркала интерферометра образуют параллелограмм. Для проведения исследований в одно из плеч интерферометра помещают ёмкость с исследуемым газом и компенсаторы.
	\subsection{Интерферометр Майкельсона}
	\begin{figure}[H]
		\centering
		\begin{tikzpicture}[>=latex']
		\filldraw[fill=gray!44!white,draw=black] (-3,-0.1) -- (-2.5,-0.1) -- (-2.5,0.1) -- (-3,0.1) -- cycle node[below] {\scriptsize{Источник}};
		\filldraw[fill=gray!44!white,draw=black] (-0.1,-2.5) -- (0.1,-2.5) -- (0.1,-3) -- (-0.1,-3) -- cycle node[left] {\scriptsize{Детектор}}; 
		\draw[blue!22!white,very thick] (-0.3,-0.3) -- (0.4,0.4);
		\draw[blue!22!white,very thick] (2.95,-0.3) -- (3.05,0.3);
		\draw[black!66!white,very thick] (2.99,-0.308) -- (3.09,0.292) node[right] {\scriptsize{Зеркало}};
		\draw[red] (3,0) -- (0.2,0.2);
		\draw[blue!22!white,very thick] (-0.3,2.5) -- (0.3,2.5);
		\draw[black!66!white,very thick] (-0.3,2.54) -- (0.3,2.54) node[above] {\scriptsize{Зеркало}};
		\draw[red] (0.2,0.2) -- (0,-2.5);
		\draw[red,->] (-2.5,0) -- (-1.5,0);
		\draw[red] (3,0) -- (0.2,0.2);
		\draw[red,->] (0.2,0.2) -- (0.1,-1.15);
		\draw[red,->] (0,0) -- (1.3,0);
		\draw[red,->] (3,0) -- (1.6,0.1);
		\draw[red] (0,-2.5) -- (0,2.5);
		\draw[red,->] (0,0) -- (0,1);
		\draw[red,->] (0,2.5) -- (0,1.75);
		\draw[red] (-2.5,0) -- (3,0);
		\draw[red,->] (0,0) -- (0,-1.5);
		\node at (1.3,0.8) {\scriptsize{Cветоделительное}};
		\node at (1.1,0.5) {\scriptsize{зеркало}};
		\end{tikzpicture}
		\caption{Схема интерферометра Майкельсона}
		\label{michelson}
	\end{figure}
Интерферометр Майкельсона (рис.~\ref{michelson}) -- двулучевой интерферометр. Конструктивно состоит из светоделительного зеркала, разделяющего входящий луч (выходящий из источника когерентного излучения) на два, которые в свою очередь, отражаются зеркалом обратно. На полупрозрачном зеркале разделённые лучи вновь направляются в одну сторону, чтобы, смешавшись на экране, образовать интерференционную картину. Анализируя её и изменяя длину одного плеча на известную величину, можно по изменению вида интерференционных полос измерить длину волны, либо, наоборот, если длина волны известна, можно определить неизвестное изменение длин плеч. Радиус когерентности изучаемого источника света или другого излучения определяет максимальную разность между плечами интерферометра.
\subsection{Применение интерферометров в научных исследованиях и технике: измерение малых смещений, изучение состояния поверхности, рефрактометрия (изменение показателя преломления).}
В силу того, что интерферометры очень чувствителны к изменению оптического хода лучей в них, при помощи интерферометров часто измеряются малые расстояния.
Рефрактометрия -- метод исследования веществ, основанный на определении показателя (коэффициента) преломления (рефракции) и некоторых его функций. Как видно выше при помощи интерферометров можно измерять показатели преломления различных веществ за счёт изменения разности хода различных световых пучков внутри интерферометра.