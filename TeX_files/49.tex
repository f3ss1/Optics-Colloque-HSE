\section{Рассеяние света, зависимость интенсивности рассеянного света от частоты(формула Рэлея) и угловая диаграмма рассеяния. Молекулярное рассеяние света. Рассеяние света в мелкодисперсных и мутных средах. }


\subsection{Pacceяние света}

\textit{Рассеяние света} - это преобразование света веществом, сопровождающееся изменением направления и проявляющееся как \textit{вынужденное} свечение, обусловленное колебаниями электронов в атомах, молекулах и ионах.

В однородной среде рассеяние может возникать, только если в среде присутствуют неоднородности. Дело в том, что вторичные волны, излученные электронами, в однородной среде взаино погашаются благодаря интерференции.

Рассеяние света в мутных средах с размерами неоднородностей $d\leq(0,1-0,2)\lambda$ называется \textit{эффектом Тиндаля}(рассеяние в мутных средах).



\subsection{Молекулярное рассеяние}

Рассеяние света наблюдается и в чистых средах без примесей. В этом случае оно называется \textit{молекулярным рассеянием}. Возникновение неоднородностей в таких средах связано с различными флуктуациями, приводящими к отклонениям показателя преломления от постоянного значения.

Если размеры неоднородностей малы по сравнению с длиной волоны, то рассеяние является \textit{упругим}: частоты исходного и рассеянного света оказываются одинаковыми.

Рассеяние света на неоднородностях, размеры которых много меньше его длины волны называется \textit{рэлеевским}.

Это явление объясняется взаимодействием электромагнитных волн с электронами. Как известно, мощность излучения диполя с дипольным моментом $p(t)$ дается формулой

\begin{equation*}
    Q = \frac{2}{3c^3}\Ddot{p}^2
\end{equation*}

\subsection{Закон Рэлея}
Пусть диэлектрическая проницаемость среды $\epsilon_0$, а диэлектрическая проницаемость неоднородного включения - $\epsilon$. Если размер неоднородности $a$ мал по сравнени. с длиной волны излучения, то электрическое поле в области включения можно считать однородным. Будем считать эту область шаром радиуса $a$. Ее индуцированный дипольный момент равен

\begin{equation*}
    p = \beta E, \beta = a^3\frac{\epsilon - \epsilon_0}{\epsilon + 2\epsilon_0}\epsilon_0,
\end{equation*}

где $E$ - электрическое поле вне шара. Пусть поле $E$ совершает гармонические колебания: $E = E_0 \cos(\omega t + \phi)$. Тогда

\begin{equation*}
    p = \beta E_0 \cos(\omega t + \phi)
\end{equation*}

Усредним по периоду колебаний и после подстановки получим

\begin{equation*}
    Q = \frac{e^4 \beta^2E_0^2}{3c^3}\omega^4
\end{equation*}

Учитывая равенство $\omega = \frac{2\pi c}{\lambda}$, получим \textit{закон Рэлея}:

\begin{equation*}
    Q = \frac{(2\pi e)^4c \beta^2E_0^2}{3\lambda^4} \sim \frac{1}{\lambda^4}
\end{equation*}


Заметим, что данная зависимость выполняется, когда частоты света находятся вдали от резонансных частот вещества: $\omega \ll \omega_0$, и зависимость диэлектрической проницаемости от длины волны излучения слабая. Это обычно имеет место при рассеянии света в прозрачных средах (в частности, в воздухе).

Для неоднородностей размерами $a > 0,2\lambda$ рэлеевское рассеяние переходит в дифракцию, причем закон Рэлея оказывается несправедливым. Если же размеры неоднородностей велики: $a \gg \lambda$, то для расчета рассеяния применимы законы геометрической оптики.

\subsection{Сечение рассеяния}

\subsubsection{Излучение колеблющегося диполя}

Пусть дипольный момент совершает колебания вдоль фиксированного направления, а его величина меняется по некоторому закону $p = p(t)$. Тогда электрическое и магнитное поля в волновой зоне даются формулами

\begin{equation*}
    E = \frac{\Ddot{p}}{c^2r}\sin\theta, H = \frac{\Ddot{p}}{c^2r} \sin\theta
\end{equation*}

Плотность потока энергии поля определяется вектором Пойнтинга:

\begin{equation*}
   S = \frac{c}{4\pi} E\times H
\end{equation*}

Отсюда после усреднения по периоду колебаний находим интенсивность излучения:

\begin{equation*}
    I = \overline{S} = \frac{\overline{\Ddot{p^2}}}{4\pi c^3 r^2}n\sin^2\theta
\end{equation*}

Эта зависимость дает угловое распределение (диаграмму направленности) дипольного излучения.




