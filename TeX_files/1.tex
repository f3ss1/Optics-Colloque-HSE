\section{Световой луч. Распространение световых лучей. Оптическая длина пути. Принцип Ферма, понятие таутохронима в оптике. Законы отражения и преломления света.}

\textbf{Световым лучом} мы будем называть некоторый конечный, но очень узкий пучок, который может существовать изолированно от других лучей.

Согласно \textbf{закону прямолинейного распространения света} световые лучи в прозрачной \textit{однородной} среде распространяются прямолинейно. Согласно \textbf{закону о независимости световых пучков} распространение каждого светового луча не зависит от того, есть ли в среде иные световые лучи, или нет. Это разумеется, не совсем корректно: тогда бы не было явления интерференции, однако в рамках геометрической оптики мы этим явлением пренебрежем.

\textbf{Оптической длиной пути} $\Delta$ между двумя точками мы назовем расстояние, на которое свет распространился бы в вакууме за время прохождения расстояния между этими двумя точками. Как известно, скорость света в вакууме есть максимальная достижимая скорость и является константой $c$, а скорость света в веществе равна:

\begin{equation}
v = \frac{c}{n}
\label{eq:n_introduct}
\end{equation}

где $n$ --- абсолютный показатель преломления среды, который и вводится, как отношение скорости света в вакууме к скорости света в этой среде.

Тогда на основании формулы (\ref{eq:n_introduct}) можем записать:

\begin{equation*}
\Delta = n l
\end{equation*} 

где $l$ --- расстояние между точками, $n$ --- абсолютный показатель преломления среды.

Указанная формула имеет обобщение на случай, когда показатель среды может зависеть от координаты, тогда:

\begin{equation}
\Delta = \int\limits_a^b n dl
\label{eq:delta_introduct}
\end{equation} 

\theornp{Принцип Ферма}{Луч движется из начальной точки в конечную по траектории, которая обеспечивает минимальную оптическую длину пути. Является \textbf{постулатом}.} 

Из принципа Ферма в случае однородной среды логичным образом вытекает закон прямолинейного распространения света, упомянутый ранее.

\theornp{Принцип таутохронизма}{Оптическая длина любого луча между двумя волновыми фронтами одна и та же.}

\theornp{Закон отражения света}{Луч падающий, луч отраженный и перпендикуляр, восстановленный из точки падения, лежат в одной полуплоскости. Угол падения равен углу отражения.}

\begin{theor}{Закон преломления света}{Луч падающий, луч преломленный и перпендикуляр, восстановленный в точке преломления, лежат в одной плоскости. Угол падения и угол преломления связаны следующим соотношением: $n_{1} \sin \alpha = n_{2} \sin \gamma$, где $n_1$ --- абсолютный показатель преломления среды, из которой луч приходит, $n_2$ --- среды, в которую луч преломляется, $\alpha$ --- угол падения, $\gamma$ --- угол преломления.}
	Пусть глаз наблюдателя находится на высоте H над поверхностью некоего водоема, а точечный источник --- на глубине $h$ в этом водоеме на расстоянии $S$ от места, где стоит наблюдатель (вдоль поверхности воды). Показатель преломления воздуха примем равным $n_{\text{возд}} = 1$, а показатель преломления воды --- $n$, известный нам.
	
	Пусть проекция отреза $BS$ на поверхность воды равна $x$. В таком случае:
	
	\begin{align*}
	BS&=\sqrt{h^{2}+x^{2}},\\
	MB&=\sqrt{H^{2}+(L-x)^{2}} \\ 
	t=\frac{SB}{v}+\frac{MB}{c}&=\left(n_{1}\sqrt{h^{2}+x^{2}}+\sqrt{H^{2}+(L-x)^{2}}\right) / c
	\end{align*}
	
	Согласно принципу наименьшего времени, полученное нами последнее выражение должно оказаться минимальным, другими словами $\dfrac{dt}{dx} = 0$ или:
	
	\begin{equation*}
	n_{1} \frac{x}{\sqrt{h^{2}+x^{2}}}-\frac{L-x}{\sqrt{H^{2}+(L-x)^{2}}}=0
	\end{equation*}
	
	\begin{figure}[H]
		\centering
		\includegraphics*[width=0.5\textwidth]{Snellius}
	\end{figure}
	
	В то же время из рисунка мы однозначно можем утверждать:
	
	\begin{align*}
	\sin \alpha &= \frac{x}{\sqrt{h^{2}+x^{2}}}\\
	\sin \gamma &= \frac{L-x}{\sqrt{H^{2}+(L-x)^{2}}}
	\end{align*}
	
	
	
	Отсюда напрямую следует факт:
	
	\begin{equation*}
	\boxed{n\sin\alpha = \sin\gamma}
	\end{equation*}
	
\end{theor}