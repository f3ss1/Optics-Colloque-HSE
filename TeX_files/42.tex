\section{Тепловое излучение. Излучательная и поглощательная способности вещества и их соотношение. Модель абсолютно черного тела. Закон Кирхгофа.}

\textbf{Поглощение света.}

\theornp{Закон Бугера --- Ламберта --- Бера}{При распространении параллельного монохроматического пучка в поглощающей среде его интенсивность изменяется по закону $$I = I_0 e^{-\alpha d },$$ где $I_0$ --- интенсивность падающего пучка, $d$ --- ширина поглощаюего образца, $\alpha$ --- коэффициент поглощения.}

Чтобы исключить влияние коэффициента отражения $R$, нужно произвести опыты с образцами разной толщины. Тогда отношение интенсивностей равно $$I_1\setminus I_2 = e^{\alpha(d_2-d_1)}, \mbox{где } \alpha = \alpha(\omega)$$

\theornp{Закон Бера}{Поглощающая способность молекулы не зависит от влияния окружающих молекул.$$\alpha = Ac, \ I = I_0e^{-Acd}, \ A = \const$$}

Если имеется ряд полос поглощения (например, в парах натрия), то вблизи них коэффициент преломления веде себя следующим образом:
$$\tilde{n} = n(1 - i\chi) \Rightarrow \alpha = \frac{4\pi}{\lambda_0}n\chi.$$
\begin{figure}[H]
	\centering
	\includegraphics*[width=\textwidth]{Na}
\end{figure}
\newpage
\textbf{Тепловое излучение.}

\theornp{Правило Прево}{При установлении теплового равновесия в изолированной системе для каждого тела должно соблюдаться равенство между количеством испускаемой и поглощаемой им в единицу времени энергии. Если тела поглощают разные количества энергии, то и испускание должно быть различно.}

\begin{figure}[H]
	\centering
	\includegraphics*[width=0.4\textwidth]{T2}
\end{figure}

a) Тарелка имеет темный узор.
б) При нагревании темные области излучают сильнее.
$$$$
\textbf{Поглощательная способность.}

$A_\omega (T)$ --- отношение поглощённого потока $\Phi'$ к падающему $\Phi$.
$$A = \frac{d\Phi'}{d\Phi}$$
$A = 0$ --- абсолютно белое тело (например, мел).
$A = 1$ --- абсолютно черное тело (уголь, сажа).


\textbf{Абсолютно черное тело.}

\textit{Абсолютно черное тело} --- физическая абстракция; тело, поглощающее всё попадающее на него электромагнитное излучение во всех диапазонах и ничего не отражающее.

Абсолютно черное тело может испускать электромагнитное излучение любой частоты и визуально изменять цвет. Его спектр определяется только его \textbf{температурой}.

\textbf{Испускание света.}

Рассмотрим тело при температуре $Т$, испускающее с элементарной площадки $d\sigma$ поток излучения $d\Phi$ с частотами в диапазоне от $\omega$ до $\omega + d\omega$.

\begin{figure}[H]
	\centering
	\includegraphics*[width=0.3\textwidth]{Rad}
\end{figure}
$E_T(\omega), E_T(\lambda)$ --- \textbf{испускательная способность}, зависит от температуры излучающего тела и не зависит от темпераруты окружающих тел.
$$d\Phi = E_T(\omega)d\omega = E_T(\lambda)d\lambda$$
$$E(\omega)=E(\lambda)\frac{\lambda^2}{2\pi c}$$
Суммарное излучение $\Phi(T) = \int_{0}^{\infty}E_T(\omega)d\omega$

\theornp{Закон Кирхгофа}{Отношение испускательной и поглощательной способностей тела не зависит от природы тела.$$\frac{E_{\omega, T}}{A_{\omega, T}}=\epsilon_{\omega, T}$$}
$\epsilon_{\omega, T}$ --- универсальная для всех тел функция температуры и частоты. Она есть не что иное, как \textit{испускательная способность абсолютно черного тела}, так как для него $$\frac{E_{\omega, T}}{A_{\omega, T}}=\frac{\epsilon_{\omega, T}}{\alpha_{\omega, T}}=\epsilon_{\omega, T}, \ \alpha = 1.$$

Рассмотрим пример абсолютно черного тела с площадкой $d\sigma$ \textit{серого тела} ($0<\alpha <1$).
\begin{figure}[H]
	\centering
	\includegraphics*[width=0.3\textwidth]{T1}
\end{figure}

Такое тело вск еще не излучает. Как устанавливается динамическое равновесие при разности излучательной способности?
Оказывается, площадка $d\sigma$ частично отражает попадающий на нее свет (т.е. не поглощает его, как абсолютно черные стенки, а отражает), чем и поддерживает равенство излучаемого потока от всех элементарных площадок.
$$Ed\sigma = A\epsilon d\sigma, \ \epsilon d\sigma \mbox{ --- падающий поток,}$$
$$d\sigma[E + (1 - A)\epsilon] = \epsilon d\sigma, \mbox{где член } (1 - A) \mbox{ возникает из-за отражения.}$$

\begin{figure}[H]
	\centering
	\includegraphics*[width=0.3\textwidth]{T3}
\end{figure}
При тепловом равновесии рисунок становится неразличим, все области излучают одинаково.