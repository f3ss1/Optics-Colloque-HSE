\section{Временная когерентность. Видность интерференционной картины. Предельная разность хода и полное число наблюдаемых интерференционных полос.}
	\subsection{Степень когерентности}
		Рассмотрим квазимонохроматический свет, с фазово-амплитудной модуляцией:
		\begin{equation*}
		E(t) = a(t)e^{i\omega_{0} t} = a_{0}(t)e^{i\delta(t)}e^{i\omega_{0} t}
		\end{equation*}
		где $a_{0}(t)$ и $\delta(t)$ -- медленно меняющиеся функции времени. Для реального света эти функуции -- случайные.
	В предположении, что приёмники света ринимают только квадраты напряжённостей (интенсивности) световых полей, усреднённых по промежуткам времени, весьма большими с периодами колебаний квазимонохроматического света и временами изменения функций вышеописанных случайных функций. Световые потоки в срелнем будем полагать стационарными.
	Квадрат поля можно представить следующим образом:
	\begin{gather*}
	\big(\Re(E)\big)^{2} = \left( \frac{E + {E}^{*}}{2}\right)^{2} = \frac{E^{2} + {E}^{*2}}{4} + \frac{E\bar{E}}{2}\\
	E^{2} + E^{*2} = 2a_{0}^{2}\cos\big(2(\omega_{0}t + \delta) \big) \text{-- бысроосциллирующая величина,}\\ \text{при усреднении выпадет, откуда}\\
	\text{интенсивность полагается:} \ I = \overline{E{E}^{*}}
	\end{gather*}
	Пусть теперь в момент времени $t$ колебания приходят в точку $P$ из источников $S_{1}$ и $S_{2}$? из которых они вышли во времена $t - \theta_{1}$ и $t - \theta_{2}$ соответственно.
	Тогда
	\begin{equation*}
	E = E(P,t) = E_{1}(t - \theta_{1}) + E_{2}(t - \theta_{2})
	\end{equation*} 
	\begin{multline*}
	I = \overline{E_{1}(t - \theta_{1})\cdot E_{1}^{*}(t - \theta_{1})} + \overline{E_{2}(t - \theta_{2})\cdot E_{2}^{*}(t - \theta_{2})} + \\ + \overline{E_{1}(t - \theta_{1})\cdot E_{2}^{*}(t - \theta_{2}) + E_{2}(t - \theta_{2})\cdot E_{1}^{*}(t - \theta_{1})}
	\end{multline*}
	В силу предположения о средней стационарности потоков первое слагаемое не зависит от $t$ и $\theta$, обозначим его $I_{1}$. Пологая $\tau$ -- отрезком времени усреднения:
	\begin{equation*}
	I_{1} = \frac{1}{\tau} \int\limits_{-\frac{\tau}{2}}^{\frac{\tau}{2}} E_{1}(t - \theta_{1})\cdot E_{1}^{*}(t - \theta_{1}) \mathrm{d} t = \frac{1}{\tau} \int\limits_{-\frac{\tau}{2}}^{\frac{\tau}{2}} E_{1}(t)\cdot E_{1}^{*}(t) \mathrm{d} t
	\end{equation*}
	Аналогично второй член 
	\begin{equation*}
	I_{2} = \frac{1}{\tau} \int\limits_{-\frac{\tau}{2}}^{\frac{\tau}{2}} E_{2}(t)\cdot E_{2}^{*}(t) \mathrm{d} t
	\end{equation*}
	Последний перекрёсный член зависит таким образом только от $\theta = \theta_{2} - \theta_{1}$, его обозначают $F_{12}(\theta)$ и называют корреляционной функцией колебаний $E_{1}(t - \theta_{1}) $ и  $ E_{2}(t - \theta_{2})$. Если $E_{1}(\cdot)$ и  $E_{2}(\cdot)$ равнs функция называется автокорреляционной (для простоты $F(\theta)$).
	Функция
	$$f_{12}(\theta) = \frac{F_{12}(\theta)}{\sqrt{I_{1}I_{2}}}$$
	называется нормированной корреляционной. Через неё запишем интенсивность в $P$.
	$$I = I_{1} + I_{2} + 2\sqrt{I_{1}I_{2}}\Re\big(f_{12}(\theta)\big) $$
	Пользуясь квазимонохроматическим приближением $E_{i} = a_{i}e^{i\omega_{0}t}, \ i = 1,2$, так что
	\begin{equation*}
	\overline{a_{1}(t) a_{2}^{*}(t - \theta)}e^{i\omega t} = \sqrt{I_{1}I_{2}}f_{12}(\theta)
	\end{equation*}
	Функция 
	\begin{equation*}
	\gamma_{12}(\theta) = f_{12}(\theta)e^{-i\omega_{0} \theta}
	\end{equation*}
	называется комплексной степенью когерентности.
	\begin{gather*}
	I = I_{1} + I_{2} + 2\sqrt{I_{1}I_{2}}\Re\big(\gamma_{12}(\theta)e^{-i\omega_{0}t}\big) \\I = I_{1} + I_{2} + 2\sqrt{I_{1}I_{2}}|\gamma_{12}(\theta)|\cos(\omega_{0}\theta + \delta)
	\end{gather*}
	Тогда
	\begin{align}
	\label{maxmin}
	I_{max} &= I_{1} + I_{2} + 2\sqrt{I_{1}I_{2}}|\gamma_{12}(\theta)| & I_{min} = I_{1} + I_{2} - 2\sqrt{I_{1}I_{2}}|\gamma_{12}(\theta)|
	\end{align}
	\subsection{Видность интерференционной картины.}
	\begin{definition}
		Видностью интерференционной картины называется величина $V$, определяемая следующим образом:
		\begin{equation}
		V = \frac{I_{max} - I_{min}}{I_{max} + I _{min}},
		\end{equation}
		где $I_{max}$ и $I_{min}$ -- интенсивности света в точках максимума и минимума интерфенренции соответственно.
	\end{definition}
	Пользуясь~(\ref{maxmin}) 
	\begin{equation*}
	V = \frac{\sqrt{I_{1}I_{2}}}{I_{1} + I_{2}}|\gamma_{12}(\theta)|
	\end{equation*}
	Если $|\gamma_{12}(\theta)|$ при всех $\theta$ равна нулю, то $V = 0$ и интерференционныйй полосы не видны, колебания называются полностью некогерентными. Когда же $|\gamma_{12}(\theta)|$ при всех $\theta$ равна единие, то когеррентность называется полной. В остальных случаях говорят о частичной некогерентности.
	\subsection{Временная когерентность}
	\begin{wrapfigure}{r}{0.5\textwidth}
		\centering
		\begin{tikzpicture}[>=latex']
		\filldraw[fill=gray!22!white, draw=black] (-1,-1.5) -- (1,-1) -- (1,1.5) -- (-1,1) -- cycle;
		\draw[->] (-2.5,0.5) -- (-1.5,0);
		\draw[->] (-2.5,1.25) -- (-1.5,0.75);
		\draw[->] (-2.5,-0.25) -- (-1.5,-0.75);
		%
		\draw[->] (-1,2) -- (0,1.5);
		\draw[->] (-1.75,1.75) -- (-0.75,1.25);
		\filldraw[fill=white, draw=black] (0.5,-0.5) circle [x radius=0.15, y radius=0.1, rotate=25] node[below] {\scriptsize $Q_{2}$};
		\filldraw[fill=white, draw=black] (-0.5,0.5) circle [x radius=0.15, y radius=0.1, rotate=25] node[below] {\scriptsize $Q_{1}$};
		\draw[->]  (0.5,-0.5) -- (2,-1.5);
		\draw[->]  (-0.5,0.5) -- (2,-1.5);
		\node at (2.2,-1.7) {\scriptsize $P$};
		\end{tikzpicture}
		\caption{Излучение от двух отверстий}
		\label{shild} 
	\end{wrapfigure}
	Пусть рассматривается когерентность одного и того же светового поля в пространственно-временных точках $R_{1}(Q_{1},t_{1})$ и $R_{2}(Q_{2},t_{2})$. Пусть пространственные точки $Q_{1}$ и $Q_{2}$ являются маленькими отверстиями в непрозрачном экране на пути светового излучения, из этих отверстий выйдет дифрагмированный свет и пусть он придёт в удалённую точку $P$ одновременно (рис.~\ref{shild}).
	
	По определению колебания в $R_{1}(Q_{1},t_{1})$ и $R_{2}(Q_{2},t_{2})$ мы называем когерентными или некогерентными, если когерентны или некогерентны соответствующие колебания в $P$. Степень когерентности $\gamma(\theta)$ определяется для них  той же величиной, что и для $R_{1}(Q_{1},t_{1})$ и $R_{2}(Q_{2},t_{2})$.
	
	Если точки $Q_{1}$ и $Q_{2}$ совпадают, но свет попадает в $P$ разными путями, то $R_{1}(Q_{1},t_{1})$ и $R_{2}(Q_{2},t_{2})$ отличаются только $t_{1}$ и $t_{2}$. В этом случае говорят о \textit{временной когерентности}. При $t_{1} = t_{2}$ степень временной когерентности равна единице,  с увелечением  разности этих времён степень когерентности убыает. Максимальное значение $|t_{2} - t_{1}|$, при котором когерентность ещё сохраняется, называется временем когерентности. Расстояние $\ell = v|t_{2} - t_{1}|$, проходимое светом за это время, называется длинной когерентности.
	   \subsection{Предельная разность хода и полное число наблюдаемых интерференционных полос.}
	   Пусть спектральный интервал излучения, создающего наблюдаемую
	   интерференционную картину, ограничен длинами волн $\lambda$ и $\lambda + \Delta \lambda$.
	   Интерференционная картина будет размываться, если максимум m-го порядка
	   для длины волны $\lambda + \Delta \lambda$ будет накладываться на максимум
	   $(m + 1)$-го порядка для длины волны $\lambda$. Тогда с учетом условия максимума
	   \begin{equation*}
	   m_{max}(\lambda + \Delta\lambda) = (m_{max} + 1) \lambda 
	   \end{equation*}
	   Откуда
		\begin{equation}
		\boxed{m_{max} = \frac{\lambda}{\Delta \lambda}}
		\end{equation}   
		С другой стороны, интерференция наблюдается до тех пор, пока разность
		хода не превышает длину когерентности $\ell$:
		\begin{equation*}
		\ell \approx m_{max}\cdot \lambda
		\end{equation*}
		Откуда предельная разность хода и длина когерентности $\delta_{max} \sim \ell$:
		\begin{equation}
		\boxed{\ell \approx \frac{\lambda^{2}}{\Delta \lambda}}
		\end{equation}