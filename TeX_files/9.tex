


\section{Линейно, циркулярно и эллиптически поляризованный свет. Поляризация естественного света. Степень поляризации. Поляризаторы. Закон Малюса}

\textit{Поляризацией} называется характеристика векторных волновых полей, описывающая поведение вектора колеблющейся величины в плоскости, перпендикулярной направлению распространения волны.

\textit{Плоскостью поляризации} называется плоскость, построенная на векторах $(E,k)$.
\subsection{Типы поляризации}
Плоская волна называется \textit{линейно поляризованной}, если ориентация плоскости поляризации не меняется во времени, то есть вектор напряженности электрического поля $E$ всегда лежит в одной и той же плоскости, содержащей вектор $k$.

В естественном (неполяризованном) свете плоскость поляризации меняется случайным образом (при этом вектор $E$ остается перпендикулярным волновому вектору $k$).

Выберем систему координат таким образом, что вектор $k$ направлен вдоль оси $z$, а вектор $E$ лежит в плоскости $(x, y)$.Пусть частота колебаний поля равна $\omega$. Тогда для компонент поля можно записать выражения

\begin{equation*}
    E_x = E_{x0}\cos(\omega t +\phi_x)
\end{equation*}


\begin{equation*}
    E_y = E_{y0}\cos(\omega t +\phi_y)
\end{equation*}

Если разность фаз этих величин равна 

\begin{equation*}
    \delta = \phi_x - \phi_y = 0, \pm\pi,
\end{equation*}


то вектор $E$ совершаеет колебания вдоль фиксированной прямой. Этот случай отвечает линейно поляризованной волне.

Если сдвиг фаз колебаний проекций отличается, то вектор $E$ совершаем вращение. В этом случае говорят, что свет имеет \textit{эллиптическую поляризацию}. Получим уравнение кривой, описываемой концом вектора $E$:

\begin{equation*}
    \left(\frac{E_x}{E_{x0}}\right)^2+\left(\frac{E_y}{E_{y0}}\right)^2 - 2\frac{E_x}{E_{x0}}\frac{E_y}{E_{y0}}\cos\delta = \sin^2\delta.
\end{equation*}

Здесь $\delta = \phi_x -\phi_y$ для сдвига фаз колебаний проекций $E_x$ и $E_y$.

Если окажется
\begin{equation*}
    \delta = \phi_x - \phi_y = \pm \frac{\pi}{2}, E_{x0} = E_{y0},
\end{equation*}

то описываемая вектором $E$ кривая есть окружность. В этом случае говорят о \textit{круговой поляризации}.

Круговая поляризация бывает двух типов - \textit{левая} и \textit{правая}. Если смотреть навстречу волновому вектору, то левой поляризации отвечает вращение вектора $E$ \textit{против часовой стрелки}. В случае правой поляризации вектор $E$ вращается \textit{по часовой стрелке}.
\subsection{Степень поляризации}

Говорят, что свет \textit{частично поляризован}, если он представляет собой смесь полностью поляризованного и естественного (неполяризованного) излучения. Пусть интенсивности этих компонент равны соответственно $I_{\text{пол}}$ и   $I_{\text{ест}}$. Тогда суммарная интенсивность составляет 
\begin{equation*}
    I_0 = I_{\text{пол}}+I_{\text{ест}}.
\end{equation*}

\textit{Степенью поляризации излучения называется отношение}


\begin{equation*}
    P = \frac{I_{\text{пол}}}{I_0} = \frac{I_{\text{пол}}}{I_{\text{пол}}+I_{\text{ест}}}.
\end{equation*}
\subsection{Поляризаторы}

\textit{Поляризатором} называют прибор, предназначенный для получения полностью или частично поляризованного оптического излучения.

Действие поляризаторов, создающих линецно поляризованный свет, основано, в частности, на следующий явлениях:

\begin{enumerate}
    \item линейный дихроизм,
    \item поляризация при отражении,
    \item поляризация при рассеянии,
    \item двойное лучепреломление.
\end{enumerate}

\subsection{Закон Малюса}

Найдем интенсивность волны, прошедшей через поляроид. Для этого разложим вектор $\textbf{E}$ на две компоненты - вдоль и поперек оси поляроида.

\begin{equation*}
    E_z = E\cos \phi, E_x = E\sin \phi
\end{equation*}

Через поляроид пройдет волна с поляризацией $E_z$. Другая же волна (с поляризацией $E_x$) полностью поглощается веществом. В итоге через поляроид проходит свет, поляризованный вдоль оси поляроида, причем его интенсивность ($I_{\text{прош}} \sim E_z^2$) равна

\begin{equation*}
    I_{\text{прош}} = I_0 \cos^2 \phi,
\end{equation*}

$I_0 \sim E^2 = E_x^2 + E_z^2$ - интенсивность волны, падающей на поляроид. Это соотношение называется законом Малюса.



