\section{Явление дифракции. Принципы Гюйгенса и Гюйгенса-Френеля. Понятие о теории дифракции Кирхгофа. Дифракция Френеля и дифракция Фраунгофера (ближняя и дальняя зоны дифракции). Волновой параметр.}

\Def{Дифракция света} -- уклонение от прямолинейного распространения, которое не является следствием отражения, преломления или изгибания света в среде. 

\Def{Принцип Гюйгенса}Каждая точка волнового фронта является источником сферических волн.

\Def{Принцип Гюйгенса-Френеля}Каждая точка поверхности, окружающей источники света, является источником вторичнных волн, распространяющихся во всех направлениях. Они когерентны, так как возбуждаются одними и теми же первичными источниками, результат их интерференции совпадает с исходным световым полем. 

\subsection{ Понятие о теории дифракции Кирхгофа.}
Основная идея Гюйгенса-Френеля в теории интерференции и дифракции волн света заключается в том, что световое возмущение в некоторой точке появляется как следствие наложения (суперпозиции) вторичных волн, которые испускаются поверхностью, расположенной между рассматриваемой точкой и источником света. Кирхгоф создал математическую форму записи принципа Гюйгенса-Френеля. Он показал, что вышеназванный принцип можно считать некоторой формой интегральной теоремы.
Интегральная теорема Кирхгофа дает возможность выразить амплитуду светового поля в точке наблюдения через интеграл по любой поверхности, которая охватывает точку наблюдения.

В теореме Кирхгофа решение однородного волнового уравнения в произвольной точке поля представлено через величину искомого параметра, его первую производную во всех точках произвольной замкнутой поверхности, которая окружает рассматриваемую точку.

Пусть волна будет монохроматической и скалярной:

\begin{equation*}
	U(r, t)=U_0(r, t)e^{(-i\omega t)}
\end{equation*}

$U_0$-комплексная амплитуда светового поля. В вакууме часть этой волны, зависящая от координат, удовлетворяет волновому уравнению Гельмгольца:

\begin{equation*}
	(\grad^2+\kappa^2)U_0=0
\end{equation*}

$\kappa=\frac{\omega}{c}$, так как само поле света удовлетворяет волновому уравнению.

Пусть V – объем, ограниченный произвольной замкнутой поверхностью S, точка А некоторая точка внутри рассматриваемого объема. Тогда одной из форм интегральной теоремы Кирхгофа – Гельмгольца:

\begin{equation*}
	U_0=\frac{1}{4\pi}\oint[{U_0\frac{\partial }{\partial n}(\frac{e^{i\kappa s}}{s})-\frac{e^{i\kappa s}}{s}\frac{\partial U_0}{\partial n}}]dS
\end{equation*}

где $\dfrac{\partial}{\partial n}$ – означает дифференцирование вдоль внутренней нормали к поверхности S. s – расстояние от точки А до точки с координатами (x,y,z).

\subsection{Дифракция Френеля. Ближняя зона дифракции}

Явления дифракции классифицируют в зависимости от расстояний источника и точки наблюдения (экрана) до препятствий, которые находятся на пути световой волны. Область дифракции, которая расположена недалеко от объекта, на котором происходит дифракция, называется ближней зоной дифракции или областью дифракции Френеля. Эта зона доходит до расстояний, с которых можно рассматривать дифракцию как фраунгоферову. В дифракционных задачах, использующих подходы Френеля нельзя пренебрегать кривизной поверхности волны, которая падает на препятствие (отверстие) и волны после дифракции. При дифракции Френеля на экране получают «дифракционное изображение» препятствия. Аналитический расчет дифракционных задач Френеля составляет существенные трудности.

\Def{Гипотеза Френеля} Для непрозрачного плоского экрана с отверстиями в качестве вспомогательной плоскости выберем неосвещенную сторону. Принцип Гюйгенса-Френеля позволяет свести задачу к определению поля на этой вспомогательной плоскости. На участках, перекрытых экраном, волновое поле 0, а на отверстиях оно определяется законами геометрической оптики. 

Недостатки теории:

\begin{enumerate} 
  \item Непонятно, как выбирать вспомогательную плоскость для неплоских экранов.
  
  \item Разрыв волнового поля на границах нарушает уравнения Максвелла.
  
  \item Если найти волновое поле во всем пространстве по принципу Гюйгенса-Френеля, оно не совпадет с исходным. Например, оно точно не обратится в нуль на задней стороне экрана.
  
  \item Гипотеза противоречит поперечности волн.
\end{enumerate}
В простейших случаях для того, чтобы установить вид картины дифракции используют метод кольцевых зон Френеля, спираль Корню. 

Особенности:

\begin{enumerate}
	\item Для оси пучка света считается, что интенсивность постоянна и равна интенсивности исходящей от источника интенсивности.\\ 2.Структура пучка света остается постоянной и задается формой отверстия. В пределах отверстия может располагаться множество зон Френеля. 
	
	\item Метод Френеля решения дифракционных задач может использоваться, когда размеры отверстий/препятствий $d\gg\lambda$, а значит, заметная интенсивность заметная при малых углах. Также, по гипотезе Френеля дифракционная картина не зависит от материала экрана.
\end{enumerate}

\subsection{Дифракция Фраунгофера. Дальняя зона дифракции}

Если расстояние между источником и экраном велико, дифракция называется дифракцией в параллельных лучах.

Область дифракции Фраунгофера простирается от бесконечности до некоторого минимального расстояния. На практике реализация дифракции Фраунгофера выполняется, если точечный источник световых волн размещают в фокусе собирающей линзы. Получившийся при этом параллельный пучок света совершает дифракцию на препятствии. Дифракционную картину наблюдают в фокальной плоскости линзы, которая размещается на пути света совершившего дифракцию или используют зрительную трубу, которую устанавливают на бесконечность. Картина дифракции является дифракционным изображением источника света. 

Особенностями дальней зоны дифракции являются: 

\begin{enumerate}
	\item Интенсивность исходной световой волны много больше, чем интенсивность света на оси пучка. 
	
	\item Интенсивность света на оси пучка уменьшается в зависимости от расстояния до источника (она обратно пропорциональна квадрату расстояния).
	
	\item Световой пучок, по мере распространения от источника, расширяется.
	
	\item В границах отверстия размещается только одна малая центральная часть зоны Френеля номер один.
\end{enumerate}	

\subsection{Волновой параметр}

\Def{Волновой параметр(число Френеля)} Определяет вид дифракции. Определяется формулой $p=\dfrac{\sqrt{\lambda z}}{b}$, где $\lambda$ - длина волны, $b$ --- размер отверстия, $z$ --- расстояние до плоскости (или до точки наблюдения):

\begin{itemize}
	\item $p<<1$ -- область геометрической оптики
	
	\item $p \sim 1$ -- область дифракции Френеля 
	
	\item $p>>1$ -- область дифракции Фраунгофера 
\end{itemize}