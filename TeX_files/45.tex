\section{Рассеяние фотонов на свободных электронах - эффект Комптона. Давление света в рамках теории фотонов.}

\textbf{Опыты Лебедева по измерению давления света.}
\begin{figure}[H]
	\centering
	\includegraphics*[width=0.5\textwidth]{Lebedev1}
	\includegraphics*[width=0.3\textwidth]{Lebedev2}
\end{figure}
В опытах Лебедева по измерению давления света использовались крутильные весы с тонкими отражающими и поглощающими крылышками, помещённые в вакуумную колбу. Крылышки освещались с помощью дуговой лампы, и по углу отклонения весов рассчитывалось световое давление.

\textbf{Экспериментальные сложности и их решение.}

\textit{Конвекционные потоки.} При несколько наклонённом положении крылышка при его нагревании излучением около него образуются возходящие потоки воздуха, действующие на него с некоторой силой, не зависящей от направлеия освещения. \textit{Решение:} помещаем установку в вакуум и проводим серию опытов, освещая лепестки с разных сторон. 

\textit{Радиометрическое действие.} С освещенной стороны лепестка молекулы газа отражаются от него с большей скоростью. Это действие зависит от стороны освещения. \textit{Решение:} сделать тонкие лепестки, чтобы температура освещенной и неосвещенной стороны были почти одинаковы.

\textbf{Давление света в рамках теории фотонов}. \newline
Энергия фотонов $E = N\hbar\omega$, $N$ --- число фотонов. Импульс фотона $p = \frac{\hbar\omega}{c}$. \newline
При поглощении фотона стенке сообщается импульс $p$ $\Rightarrow$ давление $P = Np = \frac E{c}$ \newline
При отражении сообщается импульс $2p$ $\Rightarrow$ давление $P = \frac {2E}{c}$ \newline
При коэффициенте отражения $R$ давление $P = \frac {E}{c}(1 - R +2R)= \frac {E}{c}(1 + R)$
$$\boxed{P = \frac {E}{c}(1 + R)}$$

\textbf{Эффект Комптона.} Рассматривается рентгеновское излучение с длиной волны $\lambda_0$. При прохождении волны через \textit{рассеивающее тело} --- вещество с лёгкими атомами (уголь, парафин) --- наблюдается рассеяние не только на линии рентгена $\lambda_0$, но и на смещенной линии $\lambda$.
\begin{figure}[H]
	\centering
	\includegraphics*[width=0.4\textwidth]{Kompton}
	\includegraphics*[width=0.3\textwidth]{Kompton2}
\end{figure}
Смещение линии выражается законом:
$$\Delta\lambda = \lambda - \lambda_0 = \Lambda_e(1 - \cos(\phi)),$$
гдe $\Lambda_e$ --- \textit{комптоновская длина волны электронов}. $\Lambda_e = 2,4263 \cdot 10^{-10} cm.$
\begin{figure}[H]
	\centering
	\includegraphics*[width=0.3\textwidth]{Kompton3}
\end{figure}
\textit{Закон сохранения импульса:} $$\vec{p_0} = \vec{p_e}+\vec{p} \Rightarrow p_e^2 = p_o^2 - 2pp_0\cos(\phi)+p^2,$$ $\mbox{где }\vec{p_e}\mbox{ --- импульс электрона},\vec{p_0}\mbox{ --- импульс рентгеновского фотона}.\newline$
\textit{Закон сохранения энергии:} $$\hbar\omega_0 + m_0c^2 = \hbar\omega +\frac{m_0c^2}{\sqrt{1 - \frac{v^2}{c^2}}}, \mbox{где }m_0\mbox{ --- масса электрона}, v \mbox{ --- его скорость}.$$
$$\Delta\lambda = 2\pi\frac{\hbar}{m_0c}(1-\cos{\phi}) = \Lambda_e(1-\cos{\phi}) \ \Rightarrow \Lambda_e = 2\pi\frac{\hbar}{m_0c}.$$