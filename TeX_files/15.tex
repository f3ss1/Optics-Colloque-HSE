\section{ Вращение плоскости поляризации (оптическая активность). Искусственная анизотропия оптических свойств, индуцированная механической деформацией, электрическим (эффект Керра и Поккельса) и магнитным (эффект Коттона-Муттона) полями.}
	\subsection{Естественная оптическая активность}
	\textit{Явление преломления плоскости поляризации} заключается в том, что в плоско-параллельных слоях некоторых веществ линейно поляризованный свет поворачивает плоскость своей поляризации по мере прохождения слоя. Вещества, в которых такое явление наблюдается, называются \textit{естественно-оптически активными} (в противовес магнитной оптической активности, причиной которой так же служет магнитное поле). Характерная особеннность таких веществ заключается в том, что поворот плоскости поляризации в них всегда происходит в одном и том же направлении (которое принято определять с точки зрения наблюдателя смотрящего на приближающийся свет). По этому принципу оптически активные вещества разделяют на \textit{правовращающие} и \textit{левовращающие}. Направление в котором свет проходит через этот слой не важно, и в прямом и в обратном напрвлении свет поварачивается направо от соответствующего наблюдателя. Если после прохождения слоя свет отразился и вернулся к наблюдателю через этот же слой, то направление его плоскости поляризации таким образом будет исходным. Это свойство оптической активности называется \textit{дисимметрией}.  
	Био эмпирически установил, что угол поворота плоскости поляризации $\chi$ прямо пропорционален толщине пройденного слоя $\ell$:
	\begin{equation}
	\label{chi}
	\chi = \alpha \ell
	\end{equation}
	Коэффициент $\alpha$ зависит от свойств вещества, температуры и длины волны проходящего света, причём от последней он зависит приблизительно как от обратного квадрата (то есть альфа увеличивается с уменьшением длины волны).
	Рассмотрим волну (в смысле электрического поля) $\mathbf{E} = \binom{E_{x}}{E_{y}}$
	%Рассмотри две волны поляризованные по часовой и против часовой стрелки ($\circlearrowright$ и $\circlearrowleft$ соответственно).
	\begin{align*}
	E_{x} &= A\cos(\chi)\cos(\omega t - kz) & E_{y} &= A\sin(\chi)\cos(\omega t - kz)
	\end{align*}
	Пользуясь~(\ref{chi})
	\begin{align*}
	E_{x} &= A\cos(-\alpha z)\cos(\omega t - kz) & E_{y} &= A\sin(-\alpha z)\cos(\omega t - kz)  
	\end{align*}
	Откуда
	\begin{gather}
	E_{x} = \frac{A}{2}\cos(\omega t - kz + \alpha z) + \frac{A}{2}\cos(\omega t - kz - \alpha z)\\
	E_{y} = \frac{A}{2}\cos\left(\omega t - kz + \alpha z + \frac{\pi}{2}\right) + \frac{A}{2}\cos\left(\omega t - kz - \alpha z - \frac{\pi}{2}\right)
	\end{gather}
	Введём новые обзначения
	\begin{align*}
	k^{\circlearrowright} &= k - \alpha & k^{\circlearrowleft} &= k + \alpha\\
	E_{x}^{\circlearrowright} &= \frac{A}{2}\cos(\omega t - k^{\circlearrowright} z) & E_{y}^{\circlearrowright} &= \frac{A}{2}\cos\left( \omega t - k^{\circlearrowright} z + \frac{\pi}{2}\right) \\
	E_{x}^{\circlearrowleft} &= \frac{A}{2}\cos(\omega t - k^{\circlearrowleft} z) & E_{y}^{\circlearrowleft} &= \frac{A}{2}\cos\left( \omega t - k^{\circlearrowleft}  - \frac{\pi}{2}\right)
	\end{align*}
	\begin{equation*}
	\mathbf{E} = \mathbf{E}^{\circlearrowright} + \mathbf{E}^{\circlearrowleft} = \binom{E_{x}^{\circlearrowright}}{E_{y}^{\circlearrowright}} + \binom{E_{x}^{\circlearrowleft}}{E_{y}^{\circlearrowleft}} 
	\end{equation*}
	Таким образом волна $\mathbf{E}$ раскладывается на две поляризованных по кругу в противоположных направлениях.
	Скорости этих волн:
	\begin{align*}
	v^{\circlearrowright} &= \frac{\omega}{k - \alpha} & v^{\circlearrowleft} &= \frac{\omega}{k + \alpha}
	\end{align*}
	Показатели преломления:
	\begin{align*}
	n^{\circlearrowright} &= \frac{c}{v^{\circlearrowright}} = \frac{c(k - \alpha)}{\omega} & n^{\circlearrowleft} &= \frac{c}{v^{\circlearrowleft}} = \frac{c(k + \alpha)}{\omega}
	\end{align*}
	\begin{equation}
	\label{alpha}
	\alpha = \frac{\omega}{2c}\left(n^{\circlearrowleft} - 	n^{\circlearrowright}  \right) 
	\end{equation}
	Пользуясь~(\ref{alpha}) если $\alpha > 0 $ то вращение плоскости поляризации происходит вправо, если  $\alpha < 0 $ -- влево.
	Френель показал эти соотношения опытно, позднее эти же соображения были полученны из уравнений Максвелла для материальной среды. 
	\subsection{Двойное лучепреломление}
	Явление разделения кинематической волны на две поляризованных по кругу при вращении плоскости поляризации называется \textit{круговым двойным лучепреломлением}. В общем случае двойное лучепреломление -- эффект расщепления в анизотропных средах луча света на две составляющие. Если луч света падает перпендикулярно к поверхности кристалла, то на этой поверхности он расщепляется на два луча. Первый луч продолжает распространяться прямо, и называется обыкновенным (o — ordinary), второй же отклоняется в сторону, и называется необыкновенным (e — extraordinary), причём обыкновенный и необыкновенный лучи поляризованы по-разному.
	\subsection{Искусственная анизотропия оптических свойств, индуцированная механической деформацией}
	Двойное лучепреломление можно наблюдать и в изотропных средах (аморфных телах), если подвергнуть их механическим нагрузкам.
	
	Изотропное тело, подвергнутое упругим деформациям, может стать анизотропным и изменить состояние поляризации проходящего света. Это явление носит название фотоупругости или пьезооптического эффекта. При одностороннем растяжении или сжатии тело становится подобным одноосному кристаллу с оптической осью, параллельной направлению приложенной силы. Мерой возникающей при этом оптической анизотропии служит разность показателей преломления обыкновенного и необыкновенного лучей. Опыт показывает, что эта разность пропорциональна напряжению $\sigma = \frac{\mathrm{d}F}{\mathrm{d}S}$  в данной точке тела. От этого напряжения будет зависеть разность показателей преломления:  $n_{o} - n_{e} = k\sigma$  ,  где $k$ – коэффициент пропорциональности, зависящий от свойств вещества.
	
	Поместим стеклянную пластинку $G$ между двумя поляризаторами $P_{1}$ и $P_{2}$ (рис.~\ref{mech}).
	\begin{figure}[H]
		\centering
		\begin{tikzpicture}[>=latex']
		\draw[red,->] (-3,0) -- (-2.5,0);
		\draw[red] (-3,0) -- (-2,0);
		\filldraw[fill=gray!22!white, draw=black] (-2,-0.5) -- (-2,0.5) -- (-1.5,0.5) -- (-1.5,-0.5) -- cycle node[below]{$P_{1}$};
		\draw[blue] (-2,0.5) -- (-1.5,-0.5);
		\draw[red] (-2,0) -- (-0.5,0);
		\draw[red,->] (-1.5,0) -- (-1,0);
		\filldraw[fill=blue!11!white, draw=black] (-0.5, -0.5) -- (-0.5, 0.5) -- (0.5, 0.5) -- (0.5, -0.5) -- cycle node[below]{$G$};
		\draw[->] (0,-0.75) -- (0,-0.5) node[near start, right]{\scriptsize $\vec{F}$};
		\draw[->] (0,0.75) -- (0,0.5) node[near start, right]{\scriptsize $\vec{F}$};
		\draw[red, ->] (0.5,0) -- (1,0);
		\filldraw[fill=gray!22!white, draw=black] (2,-0.5) -- (2,0.5) -- (1.5,0.5) -- (1.5,-0.5) -- cycle node[below]{$P_{2}$};
		\draw[blue] (2,0.5) -- (1.5,-0.5);
		\draw[red] (-3,0) -- (3,0);
		\draw[red,->] (2,0) -- (2.5,0);
		\end{tikzpicture}
		\caption{Искусственная анизотропия оптических свойств, индуцированная механической деформацией}
		\label{mech}
	\end{figure}
	 В отсутствие механической деформации свет через них проходить не будет. Если же стекло подвергнуть деформации, то свет может пройти, причем картина на экране получится цветная. По распределению цветных полос можно судить о распределении напряжений в стеклянной пластинке.
	\subsection{Эффект Керра и Поккельса}
	Под воздействием внешнего постоянного или переменного электрического поля в среде может наблюдаться двойное лучепреломление, вследствие изменения поляризации вещества. Пусть коэффициент преломления для обыкновенного луча равен $ n_{o}$, а для необыкновенного — $n_{e}$. Разложим разность коэффициентов преломления $n_{o} - n_{e}$, как функцию внешнего поля $E$, по степеням $E$. Если до наложения поля среда была неполяризованной и изотропной, то $n_{o}-n_{e}$ должно быть чётной функцией $E$ (при изменении направления поля эффект не должен менять знак). Значит, в разложении по степеням $E$ должны присутствовать члены лишь чётных порядков, начиная с $E^{2}$. В слабых полях членами высших порядков можно пренебречь, в результате чего
	\begin{equation}
	n_{e}-n_{o} \sim k{E}^{2}
	\end{equation}
	Эффект Керра обусловлен, главным образом, гиперполяризуемостью среды, происходящей в результате деформации электронных орбиталей атомов или молекул или вследствие переориентации последних.
	\subsection{Эффект Коттона-Муттона}Эффект Коттона — Мутона, двойное лучепреломление света в изотропном веществе, помещенном в поперечное магнитное поле (перпендикулярное световому лучу). Впервые обнаружено в коллоидных растворах Дж. Керром и (независимо от него) итальянским физиком К. Майораной в 1901. Подробно исследовано Эме Коттоном (Aime Cotton) и А. Мутоном (Н. Mouton) B 1907. Для наблюдения К.— М. э. через образец прозрачного изотропного вещества, помещенный между полюсами сильного электромагнита, пропускают монохроматический свет, линейно поляризованный в плоскости, составляющей с направлением магнитного поля угол $\frac{\pi}{4}$. В магнитном поле вещество становится оптически анизотропным (его оптическая ось параллельна магнитному полю $Н$), а проходящий свет превращается в эллиптически поляризованный, т. к. он распространяется в веществе в виде 2 волн — обыкновенной и необыкновенной, имеющих разные фазовые скорости. Разность показателей преломления обыкновенного $n_{o}$ и необыкновенного $n_{e}$ лучей, называемая величиной двойного лучепреломления, равна:
	$$n_{e} - n_{o} = kH^{2}\lambda,$$
	где Н — напряжённость магнитного поля, $k$ — зависящая от вещества константа, называемая постоянной Коттона—Мутона, $\lambda$ — длина волны света. Величина С обратно пропорциональна абсолютной температуре $Т$ и, как правило, очень мала. 