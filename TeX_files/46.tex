% Сюда нужно дописать про многоуровневые системы, но Макс говорит, что их нет в курсе.

\section{Двухуровневая система. Взаимодействие двухуровневой системы с излучением: спонтанные и вынужденные переходы. Коэффициенты Эйнштейна}

Элементарная квантовая теория теплового излучения строится на основе двухуровневой модели атома. Допускают, что у атома имеется два дискретных энергетических состояния. Одно из них называют основным, его энергию обозначим $E_0$. Второе состояние именуется возбуждённым, его энергия $E_1$ Количества атомов в каждом из данных состояний ($N_0$ и $N_1$) называют населенностями уровней основного и возбужденного.

\begin{figure}[H]\label{qr}
	\centering{\includegraphics[scale=0.25]{46_1.pdf}}
	\caption{Иллюстрация двухуровневой системы}
	\label{fig:image}
\end{figure}

Разность энергий описанных энергетических состояний равна энергии кванта света. Данный квант поглощается в том случае, если переход осуществляется с нулевого на первый уровень и излучается, если переход осуществляется с первого на уровень 0.


В соответствии с этой теорией возможно 2 вида переходов:
\begin{enumerate}
	
	\item Спонтанное излучение. Атом первоначально находится в возбужденном состоянии $E_1$. В некоторый момент времени он самопроизвольно переходит в основное состояние. При этом атом испускает фотон $\hbar \omega_{10}$. Данный процесс характеризуют вероятностью перехода в единицу времени.
	
	\item Поглощение. В данном случае. Изначально атом находится в основном состоянии и, поглощая квант света, приходит в состояние возбуждения. Вероятность данного перехода пропорциональна плотности энергии электромагнитного поля на частоте перехода и некоторому коэффициенту, который зависит от вида атома. При каждом акте поглощения количество фотонов уменьшается на один.
	
\end{enumerate}

Вышеназванные типы переходов соответствуют классической теории, но они не могут обеспечить сохранение энергетического баланса. Если вероятность спонтанного излучения зависит от внутренних свойств атома, то вероятность поглощения всегда определяется числом падающих фотонов имеющих определенную частоту. Значит, равновесие для всех частот не устанавливается. Для того чтобы устранить данное противоречие Эйнштейн предложил еще один тип перехода.


\subsection{Вынужденное излучение}

Атом переходит из возбужденного состояния в основное под действием внешнего электромагнитного поля. Количество фотонов увеличивается на один. При вынужденном излучении новый фотон невозможно отличить от фотона, который вызвал переход. Все фотоны, которые появились в результате вынужденного излучения, имеют одинаковую частоту, фазу, направление распространения и поляризацию. Вынужденное излучение когерентно.


Объектов с двумя энергетическими уровнями не существует в действительности. Существующие молекулы и атомы имеют существенно большее количество уровней энергии. Но, в условиях резонанса, если частота перехода между парой уровней в веществе близка к частоте света, обычно пренебрегают влиянием остальных уровней. В квантовой физике двухуровневая система играет важную роль, подобную роли гармонического осциллятора в классической физике. Данная модель используется в оптике для описания лазера и его взаимодействия веществом. Поэтому разговор о многоуровневой системе сводится к задаче об одноуровневой системе.

\subsection{Коэффициенты Эйнштейна}
$A_{10}$ --- вероятность спонтанного перехода атома из состояния 1 в состояние 2, $B_{10}$ ---  эйнштейновский коэффициент индуцированного излучения, $B_{01}$ --- эйнштейновский коэффициент поглощения. Тогда заселенность верхнего уровня:

\begin{equation*}
	N_1 = N_1(0)\exp(-A_{10}t)
\end{equation*}

\begin{equation*}
\left(A_{10}+B_{10} \rho_{\omega}\right) N_{1}=B_{01} \rho_{\omega} N_{0}
\end{equation*}

где $\rho(\omega)$ -- спектральная плотность. Соотношение выше показывает, что число переходов вверх равно числу переходов вниз, а также что с низкого энергетического уровня на более высокий энергетический уровень электрон может перейти, только поглотив фотон, то есть вынужденно. Спонтанно перейти на более высокий уровень атом не может, так как это вступает в противоречие с законом сохранения энергии. Переходы атомов с более высоких уровней на более низкие энергетические уровни возможны: вынужденные (вызванные внешними к атому причинами) и самопроизвольные (вызванные внутренними причинами).

Поговорим об излучении тел, закон Гиббса показывает распределение по энергетическим состояниям частиц в веществе, отсюда получаем:
\begin{equation*}
N_{J}=g_{J} N_{0} e^{-E_{J} / k T}
\end{equation*}
\begin{equation*}
\frac{g_{2}}{g_{1}} e^{-\left(E_{2}-E_{1}\right) /(k T)}=\frac{B_{12} \rho_{\omega}}{A_{21}+B_{21} \rho_{\omega}}
\end{equation*}
Замечая, что при T $\rightarrow \infty$ спектральная плотность неограниченно возрастает, замечаем симметричность коэффициента Эйнштейна у невырожденных уровней. Выражая спектральную плотность, получаем закон Планка:

\begin{equation*}
\rho_{\omega}=\frac{A_{21} g_{2}}{g_{1} B_{12}\left\{\exp \left(\dfrac{\hbar \omega_{21}}{k T}\right)-1\right\}} \qquad E_{2}-E_{1}=\hbar \omega_{21}
\end{equation*}

Тогда получаем такие соотношения ($g_i$ --- кратность вырождения i-ого уровня): 

\begin{equation*}
g_1B_{21}=g_2B_{12}, \qquad
A_{21}=B_{21} \frac{\hbar \omega_{21}^{3}}{\pi^{2} c^{3}} \frac{g_{1}}{g_{2}}
\end{equation*}

Ну и тогда число переходов из состояния 1 в 2 будет равно:

\begin{equation*}
d n_{12}=B_{12} \rho_{\omega} N_{1} d t=N_{1} \sigma_{12}(\omega) \frac{\rho_{\omega} c}{\hbar \omega_{12}} d t
\end{equation*}

где $\sigma_{12}(\omega)$ --- сечения рассеяния частиц.