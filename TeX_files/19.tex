

\section{Временная когерентность. Интерференция немонохроматических волн. Время и длина когерентности. Соотношения между временем когерентности и шириной спектрального интервала.}
\subsection{Интерференция немонохроматических волн}
\subsubsection{Корегентность}

Две волны называются \textit{когерентными}, если разность их фаз является постоянной. Когерентными являются две монохроматические волны, если только они имеют одинаковые частоты.

Если разность фаз волн меняется со временем, то эти волны называются \textit{некогерентными}.
\subsubsection{Время и длина когерентности}

Рассмотрим сложение двух волн с разными частотами $\omega_1$ и $\omega_2$.

Временем когерентности называется такое время, в течение которого разность фаз рассматриваемых волн меняется незначительно. В случае двух волн

\begin{equation*}
    A_1 = a_1 \cos(\omega_1 t + \alpha_1)
\end{equation*}

\begin{equation*}
    A_2 = a_2 \cos(\omega_2 t + \alpha_2)
\end{equation*}

разность фаз равна

\begin{equation*}
    \Delta \phi = \Delta \omega \cdot t + \Delta \alpha, \Delta \omega = \omega_1 - \omega_2, \Delta \alpha = \alpha_1 - \alpha_2
\end{equation*}

Когда сдвиг фаз составит $\Delta \omega \cdot t \sim \pi$, волны уже нельзя считать когерентными. Поэтому время когерентности определяется условием

\begin{equation*}
    \Delta \phi(t+t_{\text{ког}} - \Delta \phi(t) \sim \pi
\end{equation*}

\begin{equation*}
    t_{\text{ког}} \sim \frac{\pi}{\Delta \omega}
\end{equation*}

Переходя от частоты к длине волны по формуле $\omega = \frac{2\pi c}{\lambda}$ и считая $\Delta \omega \ll \omega$ , получим

\begin{equation*}
    t_{\text{ког}} \sim \frac{\lambda^2}{2c\Delta \lambda}
\end{equation*}


Длина когерентности - путь, проходимый волнами за время когерентности. Она составляет



\begin{equation*}
    l_{\text{ког}} = c t_{\text{ког}} \sim \frac{\lambda ^2}{2\Delta \lambda}
\end{equation*}

\subsubsection{Связь времени когерентности с шириной спектра}

Представим волновой пакет в виде суперпозиции монохроматических волн:

\begin{equation*}
    A(t) = \int_{-\infty}^{\infty} a(\omega)e^{-i\omega t} \frac{d\omega}{2\pi},\hspace{10px} a(\omega) = \int_{-\infty}^{\infty} A(t)e^{i\omega t}dt 
\end{equation*}


Пусть фурье-спектр сигнала дается выражением
\begin{equation*}
    a(\omega) = \begin{cases}
   a_0, |\omega-\omega_0| < \frac{\Delta \omega}{2}, \\
   0, |\omega-\omega_0| > \frac{\Delta \omega}{2}.
 \end{cases}
\end{equation*}

Соответствующая временная зависимость сигнала определится по первой формуле:

\begin{equation*}
    A(t) = \int_{-\infty}^{\infty} a(\omega)e^{-i\omega t} \frac{d\omega}{2\pi},\hspace{10px} \frac{a_0}{2\pi}\int_{\omega_0-\Delta\omega/2}^{\omega_0-\Delta\omega/2}e^{-i\omega t} d\omega = a_0 e^{-i\omega_0 t} F(t),
 \end{equation*}   
    
Здесь введена функция
\begin{equation*}
    F(t) = \frac{1}{2\pi it }\left(e^{i\Delta\omega t/2} - e^{-i\Delta\omega t/2}\right) = F_m\frac{\sin(\Delta\omega\cdot t/2)}{(\Delta\omega\cdot t/2)},
\end{equation*}

где $F_m(t) = \frac{\Delta \omega}{2\pi}$.

\medskip

Функция $F(t)$ обращается в первый раз в нуль при $\frac{\Delta \omega \cdot t}{2} = \pi$,  то есть в момент времени

\begin{equation*}
    t = \tau = \frac{2\pi}{\Delta \omega}.
\end{equation*}

Эта величина есть характерное время существования волнового пакета - время когерентности.
\subsection{Влияние немонохроматичности на наблюдаемое число интерференцционных полос}

Сигнал называется квазимонозроматическим, если его можно представить в виде

\begin{equation*}
    A(t) = a(t)\cos(\omega_0t+\phi(t)),
\end{equation*}


где амплитуда a(t) и фаза $\phi(t)$ - медленно меняющиеся функции.

\bigskip

Можно наблюдать интерференционную картину от квазимонозроматического источника. Действительно, свет, испускаемый таким источником, представляет собой суперпозицию монохроматических волн. Каждую из них можно расщепить на две волны(например, с помощью схемы Юнга), тогда получаемая в конце пара когерентных волн уже создает интерференционную картину.

Интерференционная каартина от всего спектра получается наложением интерференционных картин от отдельных компонент спектра. Однако, период отдельных картин зависит от соответствующей длины волны, поэтому положение максимумов и минимумов различно для разных компонент спектра. Это ограничивает общее число наблюдаемых полос от источника.

\bigskip

Рассмотрим компоненту спектра с длиной волны $\lambda$. Разность хода в опыте Юнга составляет
\begin{equation*}
    \delta \approx \frac{xd}{L}
\end{equation*}

Максимумы интерференционной картины наблюдаются в точках, для которыз $\delta = m\lambda$

\begin{equation*}
    x_m^{(max)} = \frac{\lambda L}{d}m.
\end{equation*}

Таким рьпащрм, ширина интерференционной полосы равна

\begin{equation*}
    \Delta x = \frac{\lambda L}{d}
\end{equation*}

Учтем, что источник создает немонохроматический свет в спектральном диапазоне $\Delta \lambda$. Для такого света длина когерентности есть

\begin{equation*}
    l_{\text{ког}} \sim \frac{\lambda^2}{\Delta \lambda}.
\end{equation*}

Максимальная разность хода лучей $\delta$, при которой они еще могут считаться когерентными, не должна превышать $l_{\text{ког}}$:

\begin{equation*}
    \delta < l_{\text{ког}}.
\end{equation*}

Отсюда наибольший порядок интерференции:

\begin{equation*}
    \delta_{max} = m_{max} \lambda \sim \frac{\lambda^2}{\Delta \lambda}  \Rightarrow m_{max} \sim \frac{\lambda}{\Delta\lambda}.
\end{equation*}

Тогда максимальное число наблюдаемых полос интерференции составляет

\begin{equation*}
    2m_{max}+1 \sim \frac{2\lambda}{\Delta\lambda}.
\end{equation*}

